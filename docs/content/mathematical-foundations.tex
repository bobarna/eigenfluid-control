\chapter{Mathematical Foundations}
\label{sec:mathematicalFoundations}
This chapter gives a short overview of the mathematical foundation for the
techniques we discuss later on, while also establishing the notation used in
later sections.

In the following,
\begin{equation*}
    \vb{i} = \mqty( 1 \\ 0 \\ 0 )\quad
    \vb{j} = \mqty( 0 \\ 1 \\ 0 )\quad
    \vb{k} = \mqty( 0 \\ 0 \\ 1 )
\end{equation*}
will denote the canonical basis vectors.

\section{Basic Notation}

$\vb{x} \in \mathbb{R}^n$ is considered a column-matrix, i.e. $\mathbb{R}^n
= \mathbb{R}^{n \times 1}$. This also means that $\vb{x}^T$ (the transpose of
$\vb{x}$) is a row-matrix.

We denote the scalar components of a vector $\vb{x}\in \mathbb{R}^{n}$ with $(x_1,
x_2 \dots x_n)$. We will also use $\vb{x} = (x,y,z)$ interchangably, when
$\vb{x}$ denotes a position in 3D, and $\vb{x} = (x,y)$ in 2D.

Bold uppercase letters denote matrices: $\vb{A}\in \mathbb{R}^{n \times m}$, and
its elements are indexed with $\vb{A}_ij$

A function $f(x_1, \dots x_n)$ is a scalar-valued function $\mathbb{R}^n \to
\mathbb{R}$.  When $\vb{f}: \mathbb{R}^n \to \mathbb{R}^m$ is a vector field, we
will denote it as 

$$\vb{f}(\vb{x}) = \vb{f}(x_1,\dots x_n) =(f_1(\vb{x}), \dots, f_m(\vb{x}))$$

Keeping with the conventions of fluid mechanics literature, we use the letter
$\vb{u}$ to denote the velocity of a fluid. It is hard to say where this notation
came from, but it fits another convention to call the three components of 3D
velocity $(u,v,w)$ (dropping $w$ for the 2D case).

\section{Multivariable Calculus}
\subsection*{Gradient}
The gradient $\grad$ is a generalization of the derivative to multiple
dimensions. The symbol $\grad$ is called \textit{nabla}, and typically denotes
taking partial derivatives along all spatial dimensions. 

In three dimensions, 
$$\grad f(x,y) = \mqty(\pdv*{f}{x} \\ \pdv*{f}{y} \\ \pdv*{f}{z}),$$ 
and in two dimensions,
$$\grad f(x,y) = \mqty(\pdv*{f}{x} \\ \pdv*{f}{y}).$$

It can be helpful to think of the gradient operator as a symbolic
vector, e.g. in three dimensions:
$$\grad = \mqty(\pdv*{x} \\ \pdv*{y} \\ \pdv*{z}).$$ 

We can also take gradient of vector-valued functions, which results in a matrix.
When $\vb{f}(x_1,\dots x_n) = (f_1, \dots, f_n)$, i.e. $\vb{f}: \mathbb{R}^n \to
\mathbb{R}^n$, mapping the $n$ dimensional euclidean space onto itself, it is
often called the \textit{Jacobian (matrix)} of $\vb{f}$:

$$\grad \vb{f} = \mqty[\grad^T{f_1}\\\grad^T{f_2}\\\vdots\\\grad^T f_n] = 
\mqty[
\pdv{f_1}{x_1} & \pdv{f_1}{x_2} & \dots & \pdv{f_1}{x_n}\\
\pdv{f_2}{x_1} & \pdv{f_2}{x_2} & \dots & \pdv{f_2}{x_n}\\
\vdots         &                & \ddots & \vdots \\
\pdv{f_n}{x_1} & \pdv{f_n}{x_2} & \dots & \pdv{f_n}{x_n}
].$$

\subsection*{Divergence}
The divergence operator measures how much the values of a vector field are
converging or diverging at any point in space. In three dimensions:
$$\div{\vb{u}(x,y,z)} = \div(u(\vb{x}), v(\vb{x}), w(\vb{x})) = 
\pdv{u}{x}+\pdv{v}{y}+\pdv{w}{z}.$$

Note that the input is a vector, and the output is a scalar, i.e. $\vb{u}:
\mathbb{R}^3 \to \mathbb{R}^3, \div{\vb{u}}: \mathbb{R}^3\to\mathbb{R}$).
Heuristically, in the case of a fluid velocity field $\vb{u}$, this translates
to a measure of whether a given point acts as a source, or a sink, i.e. whether
particles are created or lost in that infinitesimal region. (Later on, we will
come back to the notion of a divergence-free fluid.)

\subsection*{Curl}
The curl operator measures how much a vector field is rotating around any point. 
In three dimensions, it is given by the vector
$$\curl{\vb{u}(x,y,z) = 
    \mqty(\pdv*{x} \\ \pdv*{y} \\ \pdv*{z}) \cross 
    \mqty(u(x,y,z) \\ v(x,y,z) \\ w(x,y,z))
= \mqty|
    \vb{i}   & \vb{j}   & \vb{k}   \\
    \pdv*{x} & \pdv*{y} & \pdv*{z} \\
    u        & v        & w
|} = \mqty(
\pdv*{w}{y} - \pdv*{v}{z} \\
\pdv*{u}{z} - \pdv*{w}{x} \\
\pdv*{v}{x} - \pdv*{u}{y}
).$$

We will reduce this formula to two dimension by taking the third component of
the expression above, as if we were looking at the three-dimensional vector
field $(u,v,0)$. Thus, the two-dimensional curl is a scalar:
$$\curl{\vb{u}(x,y)} = 
    \mqty(\pdv*{x} \\ \pdv*{y}) \cross 
    \mqty(u(x,y) \\ v(x,y))
    = \pdv{v}{x} - \pdv{u}{y}.$$

\subsection*{Material Derivative}
For a velocity $\vb{u}(t,x,y,z) = \mqty(u \\ v \\ w)$, 
we define the material derivative as 
$$\dv{\vb{u}}{t} = \pdv{\vb{u}}{t} + \qty(\vb{u}\vdot\grad)\vb{u} ,$$

a special case of the total derivative. Keeping in mind that $x, y, z$ depend on
the time $t$ themselves (i.e. $\vb{u}(t, x(t), y(t), z(t))$), and utilizing the
chain rule, we can arrive on the above definition by taking the total derivative
of $\vb{u}(t, x(t), y(t), z(t))$:
\begin{alignat*}{2}
    \dv{\vb{u}}{t} &= \pdv{\vb{u}}{t}\dv{t}{t} 
                    + \pdv{\vb{u}}{x}\dv{x}{t} 
                    + \pdv{\vb{u}}{y}\dv{y}{t} 
                    + \pdv{\vb{u}}{z}\dv{z}{t} \\
                    &= \pdv{\vb{u}}{t} 1 \quad
                    + \pdv{\vb{u}}{x} u \quad
                    + \pdv{\vb{u}}{y} v \quad
                    + \pdv{\vb{u}}{z} w \\
                    &= \pdv{\vb{u}}{t} 1 \quad
                    + u \pdv{\vb{u}}{x} \quad
                    + v \pdv{\vb{u}}{y} \quad
                    + w \pdv{\vb{u}}{z} \\
                    &= \pdv{\vb{u}}{t}
                    \qty(
                        \vb{u}
                        \vdot
                        \mqty[ \pdv*{x} \\ \pdv*{y} \\ \pdv*{z} ]
                    ) \vdot \vb{u} \\
                    &= \pdv{\vb{u}}{t}
                    + \qty(\vb{u}\vdot\grad)\vdot\vb{u}.
\end{alignat*}

\subsection*{Laplacian}
The Laplacian operator is the divergence of the gradient of a scalar function
$f$. In general, for $f(\vb{x}): \mathbb{R}^n \to \mathbb{R}$, it is given by

$$\Delta f = \laplacian{f} = \grad \vdot \grad f = \sum_{i=1}^n
\pdv[2]{f}{x_i}.$$

In three dimensions, this reduces to 
$$\grad \vdot \grad f = \pdv[2]{f}{x} + \pdv[2]{f}{y} + \pdv[2]{f}{z},$$

and in two dimensions,

$$ \grad \vdot \grad f = \pdv[2]{f}{x} + \pdv[2]{f}{y}.$$

\subsection*{Vector Laplacian}
\label{section:vector-laplacian}
The Laplacian can also be applied to vector (or even matrix) fields, and the
result is simply the Laplacian of each component.

Essentially, the vector Laplacian is what we have been building towards so far,
as this operator is going to be the cornerstone of the eigenfluids simulation
technique.\todo{link to section} As such, we will show some important properties
of this operator. We will return to these in later sections.\todo{link to
section}

The vector Laplacian of a vector field $\vb{f}$ is defined as

$$\underbrace{\Delta \vb{f}}_{vector Laplacian} = 
\underbrace{\grad(\div{\vb{f}})}_{\text{gradient of the divergence}} - 
\underbrace{\curl(\curl{\vb{f}})}_{\text{curl of curl = curl}^2} =
    \text{grad(div(f))} - \underbrace{\text{curl(curl(f))}}_{\text{curl}^2(f)}$$

In Cartesian coordinates, the vector Laplacian simplifies to taking the
Laplacian of each component:

\begin{equation}
    \Delta\vb{f}(x, y, z) = (\grad\vdot\grad)\vb{f} = 
    \mqty(\Delta f_1 \\ \Delta f_2 \\ \Delta f_3) =
    \mqty(
        \pdv*[2]{f_1}{x} + \pdv*[2]{f_1}{y} + \pdv*[2]{f_1}{z}\\
        \pdv*[2]{f_2}{x} + \pdv*[2]{f_2}{y} + \pdv*[2]{f_2}{z}\\
        \pdv*[2]{f_3}{x} + \pdv*[2]{f_3}{y} + \pdv*[2]{f_3}{z}
    ).
\end{equation}

\todo{Look into this in a bit more depth. Which one is the definition, what is
the proper wording? (Anal course notes mention $\Delta$ only briefly, and
doesn't mention the vector Laplacian at all.)}

\subsection*{Integrating Multivariable Functions}
\todo{Write this. What's necessary?}


