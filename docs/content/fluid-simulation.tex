\chapter{Fluid Simulation}
Simulating convincing fluid dynamics is a continuing challenge of computer
graphics, especially when considering real-time applications. There are multiple
established ways to simulate fluids, the most widespread being Eulerian (i.e.
grid-based) and Lagrangian (i.e. particle-based) methods. Here, we restrict
ourselves to discussing Eulerian methods, and also introduce the Laplacian
Eigenfunction method (\cite{dewitt}).

The dynamics of fluids are governed by the Navier-Stokes Equations:
\begin{equation}\label{eq:NSE}
    \pdv{\vb{u}}{t} + (\vb{u}\cdot\grad)\vb{u}
    = -\frac{1}{\rho}\grad{p} + \nu\grad^2\vb{u} + \vb{f},
\end{equation}

where $\vb{u}$ is the velocity of the fluid, $\rho$ is the density, $p$ is the
scalar pressure field, $\nu$ is the viscosity constant, and $\vb{f}$ denotes
external forces. For incompressible fluids, the divergence-freeness also has to
hold, i.e. $\div{\vb{u}} = 0$

Even though equation~\eqref{eq:NSE} already describes the evolution of the fluid
as a \acf{PDE}, it is too complex for simply stepping it forward in time with
Euler steps. Instead, a technique called \textit{operator splitting} is applied
for numerical simulations, where each term is treated individually, and their
effect is combined together to fully approximate the original equation. We give
a short overview of each term to get a general understanding of fluid
simulation, first treating the problem in an Eulerian way (i.e.  sampling
$\vb{u}$ on a grid), building up our way towards the Laplacian Eigenfunction
method discussed in section~\ref{section:laplacian-eigenfluids}.  For a more
complete overview of established fluid simulation techniques, see
\cite{FluidNotes} and \cite{BridsonFluid}.

Equation~\eqref{eq:NSE} is usually split by separating out the advection part,
the external force part, and the pressure/incompressibility part. When viscosity
is important, that can also be separated. This means, we work out methods for
solving these simpler equations:
\begin{align*}
    \dv{q}{t} &= 0              \qq{(advection)}\\
    \pdv{\vb{u}}{t} &= \vb{p}   \qq{(external forces)}\\
    \pdv{\vb{u}}{t} + \frac{1}{\rho}\grad p &= 0 \\
    \qq{such that}\div{\vb{u}} &= 0.
                                \qq{(pressure, enforcing incompressibility)}
\end{align*}

A generic quantity $q$ is used in the advection equation, because as we also
show later on in our experiments, we may be interested in advecting other field
quantities than just the velocity $\vb{u}$. For the advection part, we develop
an algorithm called $\text{Advect}(\vb{u}, \Delta t, q)$: it advects quantity
$q$ through the velocity field $\vb{u}$ for a time interval $\Delta t$. 

For the external forces, any traditional numerical integration approach, such as
forward Euler can be used: $\vb{u}^{t+1} = \vb{u}^t + \Delta t \vb{f}$.

For calculating the pressure, an algorithm $\text{Project}(\Delta t, \vb{u})$
calculates and applies just the right amount of pressure to the velocity field
to make it divergence-free, and also enforce any solid wall boundary
conditions. The term "project" comes from the fact that the algorithm
essentially projects $\vb{u}$ to the closest divergence-free velocity field, and
interpreting the difference between these two fields as a pressure resulting
from "particles" being too close together. 

The order in which these algorithms are being applied matters a lot, as the
advection must be done on a divergence-free field, which means the output of
$Project$. Putting all of these together, we arrive at a basic fluid simulation
algorithm:

\begin{algorithmic}
    \State $\vb{u}^0 \gets \text{an initial divergence-free velocity field}$
    \For{$t = 0, 1, 2 \dots $}
        \State $\Delta t 
            \gets \text{a suitable timestep to go from $t_n$ to $t_{n+1}$}$
        \State $\widetilde{\vb{u}} 
            \gets \text{Advect}(\vb{u}^n,\Delta t, \vb{u}^n)$
        \State $\widetilde{\vb{u}} 
            \gets \widetilde{\vb{u}}^n + \Delta t \vb{f}$
        \State $\vb{u}^{n+1}
            \gets \text{Project}(\vb{u}^n,\Delta t, \vb{u}^n)$
            \EndFor \Comment{$[\vb{u}^0, \dots, \vb{u}^t]$ is the simulated
            fluid flow for $t$ timesteps.}
\end{algorithmic}




\section{The Laplacian Eigenfunction Method}
\label{section:laplacian-eigenfluids}
\subsection{Preliminaries}
\todo{Write Preliminaries}
We now summarize the method of using Laplacian eigenfunctions for fluid
simulation, introduced by \citet{dewitt}.


\subsection*{Laplacian Eigenfunctions as Basis Fields}
We express the velocity field $\vb{u}$ of a fluid on a domain $D$ as
$$\vb{u}(\vb{x})=\sum_i^N w_i \Phi_i(\vb{x}),$$
where the $N$ scalar elements of $\vb{w}$ are called basis coefficients, and
${\Phi_i}$ are basis functions.

As our basis functions, we choose the eigenfunctions of the vector Laplacian
operator (see Section~\ref{section:vector-laplacian}). If we further require our
basis fields $\Phi_{\vb{k}}$ to be divergence-free, and to satisfy a free slip
boundary condition, $\Phi_{\vb{k}}$ is fully characterized by

\begin{align*}
\nabla^2 \Phi_{\textbf{k}} &= \lambda_{\textbf{k}}\Phi_{\textbf{k}} \\
\nabla \cdot \Phi_{\textbf{k}} &= 0 \\
\Phi_{\textbf{k}} \cdot \textbf{n} &= 0 \text{ at } \partial D,
\end{align*}
where $\textbf{n}$ is the normal vector at the boundary $\partial D$.

Thus, on the two dimensional $D \in [0, \pi] \times [0, \pi]$ square domain, the
two scalar components of the $\Phi_{\vb{k}} =(\Phi_{\vb{k}, x},
\Phi_{\vb{k},y})$ functions take the form 

\begin{align*}
\Phi_{\textbf{k},x}(x, y) &= -\lambda_k^{-1} 
    ( k_2 \sin(k_1 x) \cos(k_2 y) ) \\
\Phi_{\textbf{k},y}(x, y) &= -\lambda_k^{-1}
    ( -k_1 \cos(k_1 x) \sin(k_2 y) ).
\end{align*}

$\textbf{k} = (k_1, k_2) \in \mathbb{Z}^2$ is known as the *vector wave
number*, with eigenvalue $\lambda_k = -(k_1^2 + k_2^2)$. 
The function is scaled by the negative inverse of its eigenvalue, 
$-\lambda_k^{-1} = 1/(k_1^2 + k_2^2)$.

\subsection*{Dynamics}
\todo{Write this section}
The vorticity formulation of the Navier-Stokes equation is
\begin{equation}\label{eq:NSE-vorticity}
    \dot{\bf{\omega}} = \text{Adv}(\bf{u}, \bf{\omega}) + \nu \Delta\bf{\omega}
    + \text{curl}(\bf{f})
\end{equation}
where $\bf{\omega} = \curl{\bf{u}}$ and $\bf{f}$ are external forces.

\subsubsection*{Precomputation of the Advection Dynamics}
\todo{}





















