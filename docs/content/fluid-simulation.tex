\chapter{Fluid Simulation}
The simulation of fluids is a long-standing challenge of computer graphics,
especially when considering real-time applications. There are multiple ways to
simulate fluids, the two most widespread being Eulerian (i.e. grid-based) and
Lagrangian (i.e. particle-based) methods. Here, we will restrict ourselves to
discussing Eulerian methods, but will also introduce the Laplacian Eigenfunction
method (\cite{dewitt}).

The dynamics of fluids are governed by the Navier-Stokes Equations:
\begin{equation}
    \label{eq:NSE}
    \begin{aligned}
        \pdv{\vb{u}}{t} = -\vb{u}\cdot\grad{\vb{u}} + 
        \nu \grad\cdot\grad \vb{u} - \grad{p} + \vb{f}
    \end{aligned}
\end{equation}

\todo{Write about simulating the NS equations, from an Eulerian simulation
perspective
- We are also advecting some marker density (add equation for that too)}
\section{The Laplacian Eigenfunction Method}
\subsection{Preliminaries}
\todo{Write Preliminaries}
We now summarize the method of using Laplacian eigenfunctions for fluid
simulation, introduced by \citet{dewitt}.

\subsection*{Laplacian Eigenfunctions as Basis Fields}
We express the velocity field $\vb{u}$ of a fluid on a domain $D$ as
$$\vb{u}(\vb{x})=\sum_i^N w_i \Phi_i(\vb{x}),$$
where the $N$ scalar elements of $\vb{w}$ are called basis coefficients, and
${\Phi_i}$ are basis functions.

As our basis functions, we choose the eigenfunctions of the vector Laplacian
operator (see Section~\ref{section:vector-laplacian}). If we further require our
basis fields $\Phi_{\vb{k}}$ to be divergence-free, and to satisfy a free slip
boundary condition, $\Phi_{\vb{k}}$ is fully characterized by

\begin{align*}
\nabla^2 \Phi_{\textbf{k}} &= \lambda_{\textbf{k}}\Phi_{\textbf{k}} \\
\nabla \cdot \Phi_{\textbf{k}} &= 0 \\
\Phi_{\textbf{k}} \cdot \textbf{n} &= 0 \text{ at } \partial D,
\end{align*}
where $\textbf{n}$ is the normal vector at the boundary $\partial D$.

Thus, on the two dimensional $D \in [0, \pi] \times [0, \pi]$ square domain, the
two scalar components of the $\Phi_{\vb{k}} =(\Phi_{\vb{k}, x},
\Phi_{\vb{k},y})$ functions take the form 

\begin{align}
\Phi_{\textbf{k},x}(x, y) &= -\lambda_k^{-1} 
    ( k_2 \sin(k_1 x) \cos(k_2 y) ) \\
\Phi_{\textbf{k},y}(x, y) &= -\lambda_k^{-1}
    ( -k_1 \cos(k_1 x) \sin(k_2 y) ).
\end{align}

$\textbf{k} = (k_1, k_2) \in \mathbb{Z}^2$ is known as the *vector wave
number*, with eigenvalue $\lambda_k = -(k_1^2 + k_2^2)$. 
The function is scaled by the negative inverse of its eigenvalue, 
$-\lambda_k^{-1} = 1/(k_1^2 + k_2^2)$.

\subsection*{Dynamics}
\todo{Write this section}
The vorticity formulation of the Navier-Stokes equation is
\begin{equation}\label{eq:NSE-vorticity}
    \dot{\bf{\omega}} = \text{Adv}(\bf{u}, \bf{\omega}) + \nu \Delta\bf{\omega}
    + \text{curl}(\bf{f})
\end{equation}
where $\bf{\omega} = \curl{\bf{u}}$ and $\bf{f}$ are external forces.

\subsubsection*{Precomputation of the Advection Dynamics}
\todo{}





















