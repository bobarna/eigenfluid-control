\chapter{Discussion and Future Work}\label{chapter:discussion}
In this work, after assessing established techniques and current research
advancements in the related fields, we introduced a novel approach to control
shape transitions by using the gradients of a fluid simulation technique based
on the eigenfunctions of the vector Laplacian operator. 

Owing to the reduced-order nature of the approach, we achieved speed-ups that
usually result in convergence times of minutes even in the case of more advanced
setups (and sub-minute, or seconds in the more straight-forward ones).

At multiple points while connecting different areas to form our proposed
solution, we resorted to baseline methods. Moving forward, our method could
benefit from incorporating a number of state-of-the-art solutions.

Although not a silver bullet, we believe that this approach compliments and
connects existing techniques in a new and exciting way, offering a fresh
perspective on thinking about Neural Networks as universal function
approximators. In the last part of our thesis, we consider some of the possible
future reasearch directions. 

\subsection*{Generalizing to 3D}
All of the introduced method generalize to 3D in a very straightforward way. As
shown by \cite{scalable-eigenfluids}, the Laplacian Eigenfluids technique is
a viable simulation for three dimensional incompressible fluid flow. The
exponential increase of simulation variables is a problem not only in forward
simulations, but especially when computing gradients for optimizing. 

\subsection*{General Improvements to the NN}
After introducing a simple training process, and purposefully keeping our
architecture simple, a number of improvements from the continuously expanding
literature on \ac{DL} and \ac{AI} techniques could be incorporated to improve
our solution.

\subsection*{Improving the Loss Function}
The loss function for the shape transition problem could also be improved in
a number of ways. In our solution, we estimate the trajectory as a linear
interpolation between start and end positions. Recalculating the trajectory
based on the actual path taken by applying the control forces could potentially
lead to more natural transition paths.

Moving further, our solution could also be improved by implementing
predictor-scheme as introduced by \cite{holl2019pdecontrol}.

\subsection*{Point Sampling}
In general, estimating functions by sampling discrete points fits into a vast
body of existing literature. The sampling strategies introduced in
section~\ref{section:sampling} could be expanded upon in a number ways, among
which improving on the correspondences between the initial and target shapes is
a noteworthy option.


