\chapter{Discussion and Future Work}

\todo{Write discussion, assess what we achieved.}

\subsection*{Generalizing to 3D}
All of the introduced method generalize to 3D in a very straightforward way. As
shown by \cite{scalable-eigenfluids}, the laplacian Eigenfluids technique is
a viable simulation for three dimensional incompressible fluid flow. The
exponential increase of simulation variables is a problem not only in forward
simulations, but especially when computing gradient for optimizing. 

\subsection*{General Improvements to the NN}
After introducing a simple training process, and purposefully keeping our
architecture simple, a number of improvements from the continuously expanding
literature on \ac{DL} and \ac{AI} techniques could be incorporated to improve
our solution.

\subsection*{Improving the Loss Function}
The loss function for the shape transition problem could also be improved in
a number of ways. In our solution, we estimate the trajectory as a linear
interpolation between start and end positions. \

\subsection*{Point Sampling}
In general, estimating functions by sampling discrete points fits into a vast
body of existing literature. The sampling strategies introduced in
section~\ref{section:sampling} could be expanded upon so that advecting the
discrete sample points approximates the natural advection trajectory of the
initial shape as defined by a scalar marker density function.

\todo{Improve upon nyakatekertség of the above.}

