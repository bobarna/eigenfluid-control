\chapter{Controlling Laplacian Eigenfluids}
Many real world applications require us to optimize for some parameters of
a physics-based problem. A toy problem would be to optimize for some initial
angle and velocity of a projectile to hit some target \todo{Refer back to the
learning to throw example}. As a more involved example, \cite{MinDrag} address
finding the best shape of a vehicle to minimize its drag. These kinds of
inverse problems have been around for quite some time in engineering
applications.

Building on all of the previously introduced ideas, we now introduce our
investigation into the use of eigenfluids in fluid control problems, making use
of their explicit closed form description of a velocity field (\todo{link to
equation}). Our main proposition is to achieve reduced-order modeling-like
speed-increase: in lieu of representing the fluid as a grid, we reconstruct the
velocity field only at discrete points in space, while simulating the dynamics
of the fluid in a reduced dimensional space.\todo{Link to eq.}

\todo{Reformulate problem as model reduction? 
\url{http://physicsbaseddeeplearning.com/others-timeseries.html}
}

\todo{Is this end-to-end training? (I think it is, but let's read up a bit on
the term.) 
}

\section{Our Experiments}
In the following, we showcase different optimization scenarios, where we try out
different aspects of controlling eigenfluids via \acf{DP} gradients. (See
section~\ref{dp-loss}.)

The first couple of examples are "traditional" optimization scenarios. By
"traditional", we mean finding individual solutions to problems via gradient
descent optimization. Moving further, we look for generalized solutions to a set
of problems by training \acfp{NN}.

\subsection{Finding Initial Velocities}
To verify feasibility before moving on to more involved setups, our most
straightforward optimziation scenario is to find an initial basis coefficient
vector $\vb{w}_0 \in \mathbb{R}^N$ for an eigenfluid simulation using $N=16$
basis fields, such that when simulated for $t$ time steps, the reconstructed
$\text{Reconstruct}(\vb{w}_t) = \vb{u}_t$ velocity field will match up some
precalculated $\vb{u}^*$ velocity field.

This is a good time to also introduce some notation 

\subsection{Controlling Shape Transitions}
\label{section:controlling-shape-transitions}
\todo{Formulate problem, loss function, etc here}

\subsubsection*{Sampling}
\todo{Write about sampling strategies of points, how shapes are defined, etc.}
\subsection{Initial Velocity}
\todo{Move some parts of this under main
Section~\ref{section:controlling-shape-transitions}}

In the following, we will showcase an optimization scenario, where we try to
find an initial base coefficient vector $\vb{w}^{0}$ for an eigenfluid
simulation with $N=16$ basis fields, such that when simulated for $t$ time
steps, the reconstructed velocity field will advect a set of initial points
$\vb{p}^0 = [p^0_0\dots p^i_0]$ to line up with target positions 
$\vb{p}^{t} = [p^t_0\dots p^t_i]$. We can write this optimization problem as 

$$\arg\min_{\vb{w}} \text{Loss}(\vb{w}, \vb{p}^0, \vb{p}^t)
= \arg\min_{\vb{w}} \left| \mathcal{P}^{t}(\vb{p}^0, \vb{w})
- \vb{p}^t\right|_2^2 = \arg\min_{\vb{w}}
\sum_{i}\left|\mathcal{P}^{t}(p^0_i, \vb{w}) - p^t_i\right|_2^2, $$

where $\mathcal{P}^t(\vb{w}, \vb{p}) = \underbrace{\mathcal{P} \circ
\mathcal{P} \dots \circ \mathcal{P}(\vb{w}, \vb{p})}_{t \text{ times}}$
denotes the physical simulation of base coefficients $\vb{w}$ and 
points $\vb{p}$ in the velocity field reconstructed from $\vb{w}$. We
use a simple mean-square error (also known as $L_2$ norm) for measuring the error.

The main difficulty of this non-linear optimization problem lies in that we have
no control over the natural flow of the fluid besides supplying an initial
$\vb{w}^0$ vector.

Making use of the differentiability of our physical simulator $\mathcal{P}$, and
the multivariable chain rule for deriving the gradient of the above
$\mathcal{P}^t$ function composition, we can derive its gradient with respect to
the initial coefficients:
$$\frac{\partial \mathcal{P}^t(\vb{w},\vb{p})}{\partial \vb{w}}.$$

\todo{Where to put Gradient Descent part? Maybe later? Or put NN training here?}
\todo{Should we add an optimization chapter? (Maybe that's too much right now)}

Finally, we can simply iterate a gradient descent optimizer with learning
rate $\mu$ to find a (good enough) solution for our above minimization problem:


$$\vb{w}_{\text{better}} = \vb{w} - \mu
\frac{
    \partial\text{Loss}(\vb{w}, \vb{p}_0, \vb{p}^t)
}{
    \partial \vb{w}
}$$

\subsection{Control Force Estimation}
\todo{Write only about problem statement + optimization part for optimizing
for $f \in \mathbb{R}^{t \times N}$ force vector.}
\subsection{Neural Network Training}
\todo{Move this section a bit up in the hierarchy? Where?}
\todo{Write about generalizing the CFE to arbitrary shapes via training a NN.}
