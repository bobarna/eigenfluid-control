\chapter{Mathematical Foundations}
\label{sec:mathematicalFoundations}
This chapter gives a short overview of the mathematical foundation for the
techniques we discuss later on, while also establishing the notation used in
later sections.

In the following,
\begin{equation*}
    \vb{i} = \mqty( 1 \\ 0 \\ 0 )\quad
    \vb{j} = \mqty( 0 \\ 1 \\ 0 )\quad
    \vb{k} = \mqty( 0 \\ 0 \\ 1 )
\end{equation*}
will denote the canonical basis vectors.

\todo{Check that this holds true throughout the paper}
$\vb{x} \in \mathbb{R}^n$ is considered a column-matrix, i.e. $\mathbb{R}^n
= \mathbb{R}^{n \times 1}$. This also means that $\vb{x}^T$ (the transpose of
$\vb{x}$ is a row-matrix.

\section{Vectors, Matrices and Fields}

\section{Functions}

\section{Multivariable Calculus}

\subsection*{Gradient}
The gradient $\grad$ is a generalization of the derivative to multiple
dimensions. The symbol $\grad$ is called \textit{nabla}, and typically denotes
taking partial derivatives along all spatial dimensions. 

E.g. in 3 and 2 dimensions:
$$\grad = \mqty(\pdv*{x} \\ \pdv*{y} \\ \pdv*{z}) \qquad
\grad = \mqty(\pdv*{x} \\ \pdv*{y})$$

\subsection*{Jacobian}

\todo{Does the Jacobian matrix add value here?}

When $\vb{f}(x_1,\dots x_n) = (f_1, \dots, f_n)$, i.e. $\vb{f}: \mathbb{R}^n \to
\mathbb{R}^n$, mapping the $n$ dimensional euclidean space onto itself, the
\textit{Jacobian matrix} of $\vb{f}$ is defined as:

\todo{try out inline pdv ?}

$$\vb{J} = \mqty[\grad{f_1}\\\grad{f_2}\\\vdots\\\grad^T f_n] = 
\mqty[
\pdv{f_1}{x_1} & \pdv{f_1}{x_2} & \dots & \pdv{f_1}{x_n}\\
\pdv{f_2}{x_1} & \pdv{f_2}{x_2} & \dots & \pdv{f_2}{x_n}\\
\vdots         &                & \ddots & \vdots \\
\pdv{f_n}{x_1} & \pdv{f_n}{x_2} & \dots & \pdv{f_n}{x_n}
]$$


\subsection*{Material Derivative}
For a velocity $\vb{u}(t,x,y,z) = \mqty(u \\ v \\ w)$, 
we define the material derivative as 
$$\dv{\vb{u}}{t} = \pdv{\vb{u}}{t} + \qty(\vb{u}\vdot\grad)\vb{u} ,$$

a special case of the total derivative. Keeping in mind that $x, y, z$ depend on
the time $t$ themselves (i.e. $\vb{u}(t, x(t), y(t), z(t))$), and utilizing the
chain rule, we can arrive on the above definition by taking the total derivative
of $\vb{u}(t, x(t), y(t), z(t))$:
\begin{alignat*}{2}
    \dv{\vb{u}}{t} &= \pdv{\vb{u}}{t}\dv{t}{t} 
                    + \pdv{\vb{u}}{x}\dv{x}{t} 
                    + \pdv{\vb{u}}{y}\dv{y}{t} 
                    + \pdv{\vb{u}}{z}\dv{z}{t} \\
                    &= \pdv{\vb{u}}{t} 1 \quad
                    + \pdv{\vb{u}}{x} u \quad
                    + \pdv{\vb{u}}{y} v \quad
                    + \pdv{\vb{u}}{z} w \\
                    &= \pdv{\vb{u}}{t} 1 \quad
                    + u \pdv{\vb{u}}{x} \quad
                    + v \pdv{\vb{u}}{y} \quad
                    + w \pdv{\vb{u}}{z} \\
                    &= \pdv{\vb{u}}{t}
                    \qty(
                        \vb{u}
                        \vdot
                        \mqty[ \pdv*{x} \\ \pdv*{y} \\ \pdv*{z} ]
                    ) \vdot \vb{u} \\
                    &= \pdv{\vb{u}}{t}
                    + \qty(\vb{u}\vdot\grad)\vdot\vb{u}.
\end{alignat*}

\subsection*{Divergence}
Given $\vb{f}(x,y,z) = \mqty(f_1(x,y,z)\\f_2(x,y,z)\\f_3(x,y,z)), 
\vb{f}: \mathbb{R}^n \to \mathbb{R}^n, \div{\vb{f}}:
\mathbb{R}^n\to\mathbb{R}$.

$$\div{\vb{f}(x,y,z)} = \pdv{f_1}{x}+\pdv{f_2}{y}+\pdv{f_3}{z}$$

\subsection*{Curl}
In 3 dimensions:
$$\curl{\vb{f}(x,y,z) = 
    \mqty(\pdv*{x} \\ \pdv*{y} \\ \pdv*{z}) \cross 
    \mqty(f_1(x,y,z) \\ f_2(x,y,z) \\ f_3(x,y,z))
= \mqty|
    \vb{i}   & \vb{j}   & \vb{k}   \\
    \pdv*{x} & \pdv*{y} & \pdv*{z} \\
    f_1      & f_2      & f_3
|} = \mqty(
\pdv*{f_3}{y} - \pdv*{f_2}{z} \\
\pdv*{f_1}{z} - \pdv*{f_3}{x} \\
\pdv*{f_2}{x} - \pdv*{f_1}{y}
)$$

In 2 dimensions:
$$\curl{\vb{f}(x,y)} = 
    \mqty(\pdv*{x} \\ \pdv*{y}) \cross 
    \mqty(f_1(x,y) \\ f_2(x,y))
    = \pdv{f_2}{x} - \pdv{f_1}{y}$$

\subsection*{Laplacian}
The Laplacian operator gives the second derivatives of a function
$f(x,y,z)$:

$$\Delta f = \laplacian{f} = (\grad \vdot \grad)f 
= \mqty[\pdv*[2]{f}{x} \\ \pdv*[2]{f}{y} \\ \pdv*[2]{f}{z}]$$

\subsection*{Vector Laplacian}
Essentially, this is what we have been building towards so far, as this operator
is going to be the cornerstone of the eigenfluids simulation
technique.\todo{link to section} As such, we will show some important
properties of this operator. We will return to these in later
sections.\todo{link to section}


In general: $\vb{f}: \mathbb{R}^n \to \mathbb{R}^n, \Delta\vb{f}:
\mathbb{R}^n\to\mathbb{R}^{n \times n}$, i.e. in three dimensions given
$\vb{f}(x,y,z) = \mqty(f_1(x,y,z)\\f_2(x,y,z)\\f_3(x,y,z)), \vb{f}:
\mathbb{R}^3\to\mathbb{R}^3, \Delta \vb{f}: \mathbb{R}\to\mathbb{R}^{n\times
n}$.

\begin{equation}
    \Delta\vb{f}(x, y, z) = (\grad\vdot\grad)\vb{f} = 
    \mqty(\Delta f_1 \\ \Delta f_2 \\ \Delta f_3) =
    \mqty(
        \pdv*[2]{f_1}{x} + \pdv*[2]{f_1}{y} + \pdv*[2]{f_1}{z}\\
        \pdv*[2]{f_2}{x} + \pdv*[2]{f_2}{y} + \pdv*[2]{f_2}{z}\\
        \pdv*[2]{f_3}{x} + \pdv*[2]{f_3}{y} + \pdv*[2]{f_3}{z}
    )
\end{equation}

It can be shown that
$$\underbrace{\Delta \vb{f}}_{vector Laplacian} = 
\underbrace{\grad(\div{\vb{f}})}_{\text{gradient of the divergence}} - 
\underbrace{\curl(\curl{\vb{f}})}_{\text{curl of curl = curl}^2} =
    \text{grad(div(f))} - \underbrace{\text{curl(curl(f))}}_{\text{curl}^2(f)}$$

\todo{add proof ?}

\subsection*{Integrating Multivariable Functions}
\todo{write this}


