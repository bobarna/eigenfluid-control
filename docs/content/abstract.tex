\pagenumbering{roman}
\setcounter{page}{1}

\selecthungarian

%----------------------------------------------------------------------------
% Abstract in Hungarian
%----------------------------------------------------------------------------
\chapter*{Kivonat}\addcontentsline{toc}{chapter}{Kivonat}

Fizikai környezetünk megértése és modellezése egy régóta fennálló kihívás, ami
rengeteg tudományterületet érint az időjárás előrejelzéstől kezdve, járművek
tervezésén át egészen a számítógépes grafikáig. Fizikai rendszereket általában
parciális differenciálegyenletek segítségével írunk le, amiket meglévő numerikus
módszerekkel tudunk közelíteni. A szimuláció mellett fontos feladat lehet egy
fizikai folyamat irányítása is.  Jelen munka azt járja körbe, hogy hogyan tudjuk
folyadékok viselkedését leírni és irányítani Laplace sajátfüggvények
segítségével. Az explicit függvények általi leírásból megkapott deriváltakkal
neurális hálókat tanítunk a fizikai folyamat kívánt módon történő
befolyásolására.

\todo{Erre volt komment. Így jó?}

\vfill
\selectenglish

%----------------------------------------------------------------------------
% Abstract in English
%----------------------------------------------------------------------------
\chapter*{Abstract}\addcontentsline{toc}{chapter}{Abstract}

Understanding and modelling our environment is a great and important challenge,
spanning many disciplines from weather and climate forecast, through vehicle
design to computer graphics. Physical systems are usually described by Partial
Differential Equations (PDEs), which we can approximate using established
numerical techniques. Next to predicting outcomes, planning interactions to
control physical systems is also a long-standing problem.

In our work, we investigate the use of Laplacian Eigenfunctions to model and
control fluid flow. We make use of an explicit description of our simulation
domain to derive gradients of the physical simulation, enabling neural network
agents to learn to control the physical process to achieve desired outcomes.

\vfill
\cleardoublepage

\selectthesislanguage

\newcounter{romanPage}
\setcounter{romanPage}{\value{page}}
\stepcounter{romanPage}
