\pagenumbering{roman}
\setcounter{page}{1}

\selecthungarian

%----------------------------------------------------------------------------
% Abstract in Hungarian
%----------------------------------------------------------------------------
\chapter*{Kivonat}\addcontentsline{toc}{chapter}{Kivonat}

Fizikai környezetünk megértése és modellezése egy régóta fennálló kihívás, ami
rengeteg tudományterületet érint az időjárás előrejelzéstől kezdve, járművek
tervezésén át egészen a számítógépes grafikáig. Fizikai rendszereket általában
parciális differenciálegyenletek segítségével írunk le, amiket meglévő numerikus
módszerekkel tudunk közelíteni. A szimuláció mellett fontos feladat lehet egy
fizikai folyamat irányítása is.  

Dolgozatom központi témája, hogy hogyan tudunk gradiens-alapú optimalizálási
módszerek számára meglévő tudást átadni fizikai folyamatok működéséről.
A folyamat gradiensei a felügyelt tanításban megszokott hibafüggvény értéke
mellett arról is tudást adnak át az optimalizációnak (``ágensnek''), hogy egy
adott pillanatban hozott döntése hogyan befolyásolja nemlineáris fizikai
rendszerek lefolyását.

Több kutatási irány összekapcsolásával azt járom körbe, hogyan tudjuk folyadékok
viselkedését leírni és irányítani egy csökkentett dimenziójú módszer
segítségével. Sűrűségfüggvények advekcióját mintapontokkal közelítem, amiket
részecskeként szimulálok a folyadék sebességmezőjében. A módszer előnye, hogy
a Laplace-operátor sajátfüggvényeinek lineáris kombinációjaként  a sebességmező
zárt alakban mintavételezhető. Így a folyadékot a benne áramló anyagokkal
együtt  anélkül tudjuk szimulálni, hogy a teljes tartományt számon kellene
tartani.

Dolgozatomban különböző megközelítésekkel egyre összetettebb problémákat
modellezek. Először egyes esetek megoldására nyújtok megoldást gradiens alapú
optimalizálás segítségével, majd általánosítva a problémát neurális hálókat
tanítok be a fizikai folyamat kívánt módon történő irányítására.

\vfill
\selectenglish

%----------------------------------------------------------------------------
% Abstract in English
%----------------------------------------------------------------------------
\chapter*{Abstract}\addcontentsline{toc}{chapter}{Abstract}

Understanding and modeling our environment is a great and important challenge,
spanning many disciplines from weather and climate forecast, through vehicle
design to computer graphics. Physical systems are usually described by Partial
Differential Equations (PDEs), which we can approximate using established
numerical techniques. Next to predicting outcomes, planning interactions to
control physical systems is also a long-standing problem.

In our work, we investigate the use of Laplacian Eigenfunctions to model and
control fluid flow. We make use of an explicit description of our simulation
domain to derive gradients of the physical simulation, enabling neural network
agents to learn to control the physical process to achieve desired outcomes.

\vfill
\cleardoublepage

\selectthesislanguage

\newcounter{romanPage}
\setcounter{romanPage}{\value{page}}
\stepcounter{romanPage}
