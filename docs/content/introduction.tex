\chapter{\bevezetes}

Laying out the motivation, previous work and overall structure of the paper.

\todo{Finish introduction}

\section{Previous Work}
\subsection{Fluid Simulation}
\todo{Mention Langrangian simulation techniques too}

\subsubsection*{Eulerian Techniques}
Eulerian, i.e. grid-based fluid simulation techniques are currently employed in
most solutions. 

\todo{cite Stable Fluids as base method? What other papers to mention?}

\subsubsection*{Reduced Order Modeling of Fluids}
Dimension reduction-based techniques have been applied to fluid simulation in
multiple previous works. 

\todo{Add some basic ROM papers next to these recent ones?}

Recently, \cite{LatentSpaceSubdivision} predicted the evolution of fluid flow
via training a convolutional neural network (CNN) for spatial compression, with
another network predicting the temporal evolution in this compressed subspace.
The main novelty of \cite{LatentSpaceSubdivision} was the subdivision of the
latent space into subparts, allowing interpretability, as well as external
control of quantities such as velocity and density of the fluid.

\cite{Wiewel2019LatentSP} demonstrated that functions of an evolving physics
system can be predicted within the latent spaces of neural networks. Their
efficient encoder-decoder architecture predicted pressure fields, yielding two
orders of magnitudes faster simulation times than a traditional pressure solver.

\subsubsection*{Eigenfluids}
Instead of learning a reduced-order representation, another option is to
analytically derive the dimension reduction, and its time evolution.

\cite{dewitt} introduced a computationally efficient fluid simulation technique
to the computer graphics community. Rather than using an Eulerian grid or
Lagrangian particles, they represent velocity and vorticity using a basis of
global functions defined over the entire simulation domain. They propose the use
of Laplacian eigenfunctions. Following their method, the fluid simulation
becomes a matter of evolving basis coefficients in the space spanned by
these eigenfunctions, resulting in a speed-up characteristic of reduced-order
methods.

Following up on the work of \cite{dewitt}, multiple papers proposed improvements
to the use of Laplacian eigenfunctions for the simulation of incompressible
fluid flow. \cite{ModelReductionFluidSim} extended the technique to handle
arbitrarily-shaped domains. \cite{EigenfluidCompression} used Discrete Cosine
Transform (DCT) on the eigenfunctions for compression.
\cite{scalable-eigenfluids} improved scalability of the technique, and modified
the method to handle further types of boundary conditions.

\subsection{Differentiable Solvers}
Differentiable solvers have shown tremendous success lately for optimization
problems via gradient descent, including training neural nework agents.
(\cite{holl2019pdecontrol}, \cite{difftaichi}, \cite{warp2022})

\todo{\cite{holl2019pdecontrol}, \cite{difftaichi}, \cite{warp2022} are
frameworks, strictly speaking (although they all respectively contain relevan
diff. physics examples.}

\cite{holl2019pdecontrol} address grid-based solvers, noting them as
particularly appropriate for dense, volumetric phenomena. They put forth
$\Phi_{Flow}$, an open-source simulation toolkit built for optimization and
machine learning applications, written mostly in Python.

\subsubsection*{Physics-based Deep Learning}
Integrating physical solvers in training-based methods have been shown to
outperform previously used learning approaches.(\cite{solver-in-the-loop})

Drawing on a wide breadth of current research, \cite{pbdl} give an overview of
deep learning methods in the context of physical simulations. 

\todo{Maybe elaborate on this a bit more?}

\section{Structure}
\todo{Short description of each chapter. (Once chapters are actually written.)}

