\chapter{\bevezetes}

In this chapter, we briefly discuss previous work, and the overall structure
of our paper. 

\todo{Write about our motivation here? (Also see note at the end of the chapter:
should we mention our contribution here?)}

\section{Previous Work}
\subsection{Fluid Simulation}
Most simulation methods are based on either an Eulerian (i.e.  grid-based), or
Lagrangian (i.e. particle-based) representation of the fluid.  For an overview
of fluid simulation techniques in computer graphics, see \cite{FluidNotes} and
\cite{BridsonFluid}.  Eulerian simulation techniques are currently employed in
most solutions. For comparison with reduced order techniques, we also use
grid-based techniques in our work, for the most part as described by
\cite{StableFluids}.

\todo{Is it okay to keep this part so short?}

\subsubsection*{Reduced Order Modeling of Fluids}
Dimension reduction-based techniques have been applied to fluid simulation in
multiple previous works. \cite{Wiewel2019LatentSP} demonstrated that functions
of an evolving physics system can be predicted within the latent space of neural
networks. Their efficient encoder-decoder architecture predicted pressure
fields, yielding two orders of magnitudes faster simulation times than
a traditional pressure solver.

Recently, \cite{LatentSpaceSubdivision} predicted the
evolution of fluid flow via training a convolutional neural network (CNN) for
spatial compression, with another network predicting the temporal evolution in
this compressed subspace.  The main novelty of \cite{LatentSpaceSubdivision} was
the subdivision of the learned latent space, allowing interpretability, as well
as external control of quantities such as velocity and density of the fluid.

\subsubsection*{Eigenfluids}
Instead of \textit{learning} a reduced-order representation, another option is to
analytically derive the dimension reduction, and its time evolution.
\cite{dewitt} introduced a computationally efficient fluid simulation technique
to the computer graphics community. Rather than using an Eulerian grid or
Lagrangian particles, they represent velocity and vorticity using a basis of
global functions defined over the entire simulation domain. They propose the use
of Laplacian eigenfunctions. Following their method, the fluid simulation
becomes a matter of evolving basis coefficients in the space spanned by these
eigenfunctions, resulting in a speed-up characteristic of reduced-order methods.

Following up on the work of \cite{dewitt}, multiple papers proposed improvements
to the use of Laplacian eigenfunctions for the simulation of incompressible
fluid flow. \cite{ModelReductionFluidSim} extended the technique to handle
arbitrarily-shaped domains. \cite{EigenfluidCompression} used Discrete Cosine
Transform (DCT) on the eigenfunctions for compression.
\cite{scalable-eigenfluids} improved scalability of the technique, and modified
the method to handle further types of boundary conditions.
\cite{scalable-eigenfluids} refers to the fluid simulation technique using
Laplacian eigenfunctions as \textit{eigenfluids}, which we will also adhere to
in the following.

\todo{They actually write "Eigenfluid" (uppercase), but I like it better as
lowercase, such as "grid-based" or "particle-based".}

\subsection{Differentiable Solvers}
Differentiable solvers have shown tremendous success lately for optimization
problems via gradient descent, including training neural nework agents
(\cite{holl2019pdecontrol}, \cite{difftaichi}, \cite{warp2022}).

\todo{\cite{holl2019pdecontrol}, \cite{difftaichi}, \cite{warp2022} are
frameworks, strictly speaking (although they all respectively contain relevant
diff. physics examples. Is that OK?}

\cite{holl2019pdecontrol} address grid-based solvers, noting them as
particularly appropriate for dense, volumetric phenomena. They put forth
$\Phi_{Flow}$, an open-source simulation toolkit built for optimization and
machine learning applications, written mostly in Python.

\subsubsection*{Physics-based Deep Learning}
Despite being a topic of research for a long time (\cite{backprop}), the
interest in neural network algorithms is a relatively new phenomenon. This is
especially true for the use of learning-based methods in physical and numerical
simulations, which is a rapidly developing area of current research. Recently,
integrating physical solvers in such methods have been shown to outperform
previously used learning approaches.(\cite{solver-in-the-loop})

Drawing on a wide breadth of current research, \cite{pbdl} give an overview of
deep learning methods in the context of physical simulations. 

\todo{Maybe elaborate on this a bit more?}

\section{Structure}
\todo{Short description of each chapter. (Once chapters are actually written.)}

\todo{Write about our contribution here: using physics-based gradients of
a reduced-order simulation technique to solve optimization problems?}
