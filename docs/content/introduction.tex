\chapter{\bevezetes}

Laying out the motivation, previous work and overall structure of the paper.

\todo{Finish introduction}

\section{Previous Work}
\subsection{Fluid Simulation}
\todo{Mention Langrangian simulation techniques too}

\subsubsection*{Eulerian Techniques}
Eulerian, i.e. grid-based fluid simulation techniques are currently employed in
most solutions. 
\todo{cite Stable Fluids as base method? What other stuff to mention?}

\subsubsection*{Reduced Order Modeling of Fluids}
\todo{Write about Reduced Order Modeling (ROM) in general}
\todo{Previous work on finding Koopman operators for a linear embedding}
\todo{\url{https://github.com/wiewel/LatentSpaceSubdivision}}
\todo{\url{https://github.com/byungsook/deep-fluids}}

\subsubsection*{Eigenfluids}

\cite{dewitt} introduced a computationally efficient fluid simulation technique
to the computer graphics community. Rather than using an Eulerian grid or
Lagrangian particles, they represent velocity and vorticity using a basis of
global functions defined over the entire simulation domain. They propose the use
of Laplacian eigenfunctions. Following their method, the fluid simulation
becomes a matter of evolving basis coefficients in the space spanned by
these eigenfunctions, resulting in a speed-up characteristic of reduced-order
methods.

Multiple follow-up papers proposed improvements to the technique.

\todo{list improvements of scalable eigenfluids, etc...}

\subsection{Differentiable Solvers}
\cite{holl2019pdecontrol} address grid-based solvers, noting them as
particularly appropriate for dense, volumetric phenomena. They put forth
\textit{PhiFlow}, an open-source simulation toolkit build for optimization and
machine learning applications, written mostly in Python.

\todo{Write more about \cite{holl2019pdecontrol}.}

\subsubsection*{Physics-based Deep Learning}
Integrating physical solvers in training-based methods have been shown to
outperform previously used learning approaches.(\cite{solver-in-the-loop})

Drawing on a wide breadth of current research, \cite{pbdl} give an overview of
deep learning methods in the context of physical simulations. 

\todo{Elaborate this more. Maybe later on, when we see our own results.}


\section{Structure}
\todo{Short description of each chapter. (Once chapters are actually written.)}

